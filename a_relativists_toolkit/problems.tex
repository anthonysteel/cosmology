\documentclass[9pt]{report}
\usepackage{graphicx}
\usepackage[utf8]{inputenc}
%\usepackage[T1]{fontenc}
\usepackage{textcomp}
%\usepackage[dutch]{babel}
\usepackage{amsmath, amssymb}

\begin{document}
\title{A Relativist's Toolkit\protect\\ Problems}
\author{Anthony Steel}
\date{\today}
\maketitle
\chapter{Fundamentals}
\begin{enumerate}
  \item \textbf{The surface of a two-dimensional cone is embedded in
      three-dimensional flat space. The cone has an opening angle of
      $2\alpha$. Points on the cone which all have the same distance
      $r$ from the apex define a circle, and $\phi$ is the angle that
      runs along the circle.
    }
    \begin{enumerate}
      \item \textbf{Write down the metric of the cone, in terms of the
        coordinates $r$ and $\phi$.}
        Consider the metric for 3-space in spherical polar coordinates:
        \[
          ds^2 = dr^2 + r^2d\theta^2 + r^2\sin^2\theta d\phi^2
        \]
        On a cone, the cooridnate $\theta$ is half of the opening angle and is a
        constant implying: $\theta=\alpha$ and $d\theta=d\alpha=0$. Therefore
        the metric of the cone is:
        \[
          ds^2 = dr^2 + r^2\sin^2\alpha d\phi^2
        \]
        Shown another way, consider the paramaterization of the cone $X(r, \phi)$
        \[
          X(r, \phi) =
          \begin{bmatrix}
            r \sin\alpha \cos \phi\\
            r \sin\alpha \sin \phi \\
            r \cos \alpha
          \end{bmatrix}
        \]
        Differentiating with respect to $r$ and $\phi$ gives:
        \[
          X_{,\phi} =
          \begin{bmatrix}
            - r \sin\alpha \sin\phi \\
              r \sin\alpha \cos\phi \\
              0
          \end{bmatrix}
        \]
        and
        \[
          X_{,r} =
          \begin{bmatrix}
            \sin\alpha \cos \phi \\
            \sin \alpha \sin \phi \\
            \cos\alpha
          \end{bmatrix}
        \]
        The general form of the metric in two dimensions is:
        \[
          ds^2 = E d\phi^2 + 2F d\phi dr + G dr^2
        \]
        where:
        \[
          \begin{align}
            E &= X_{,\phi} \cdot X_{,\phi} = r\sin^2\alpha \\
            F &= X_{,\phi} \cdot X_{,r} = 0\\
            G &= X_{,r} \cdot X_{,r} = 1
          \end{align}
        \]
        therefore:
        \[
          ds^2 = dr^2 + r^2\sin^2\alpha^2 d\phi^2
        \]

      \item \textbf{Find the coordinate transformation $x(r,\phi)$,
          $y(r,\phi)$ that brings the metric into the form $ds^2
        = dx^2+dy^2$. Do these coordinates cover the entire
        two-dimensional plane?}

        Consider the differentials of the transformation $x(r, \phi)$ and
        $y(r, \phi)$:
        \[
          \begin{align}
            dx &= \frac{\partial x}{\partial r} dr + \frac{\partial x}{\partial \phi} d\phi \\
            dy &= \frac{\partial y}{\partial r} dr + \frac{\partial y}{\partial \phi} d\phi \\
          \end{align}
        \]
        Therefore:
        \[
          \begin{align}
           dx^2 + dy^2 &= \Big[ \Big(\frac{\partial x}{\partial r}\Big)^2 + \Big(\frac{\partial y}{\partial r}\Big)^2 \Big]dr^2 +
                          \Big[ \Big(\frac{\partial x}{\partial \phi}\Big)^2 + \Big(\frac{\partial y}{\partial \phi}\Big)^2\Big] d\phi^2 +
                          \Big(\frac{\partial x}{\partial r}\frac{\partial x}{\partial \phi} + \frac{\partial y}{\partial r}\frac{\partial y}{\partial \phi}\Big) dr d\phi \\
                       &= dr^2 + r^2\sin^2\alpha d\phi^2
          \end{align}
        \]
        Which gives three equations:
        \begin{equation}
          \label{eqn:first}
          \Big(\frac{\partial x}{\partial r}\Big)^2 + \Big(\frac{\partial y}{\partial r}\Big)^2 = 1
        \end{equation}
        \begin{equation}
          \label{eqn:second}
          \Big(\frac{\partial x}{\partial \phi}\Big)^2 + \Big(\frac{\partial y}{\partial \phi}\Big)^2 = r^2 \sin^2\alpha
        \end{equation}
        \begin{equation}
          \label{eqn:third}
          \Big(\frac{\partial x}{\partial r}\frac{\partial x}{\partial \phi} + \frac{\partial y}{\partial r}\frac{\partial y}{\partial \phi}\Big) = 0 \to \frac{\partial x}{\partial r}\frac{\partial x}{\partial \phi} = -\frac{\partial y}{\partial r}\frac{\partial y}{\partial \phi}
        \end{equation}
        Make the following ansatz:
        \[
          \begin{align}
            x(r, \phi) &= r\sin\phi \\
            y(r, \phi) &= r\cos\phi
          \end{align}
        \]
        \[
          \begin{align}
            \frac{\partial x}{\partial r} &= \sin\phi \\
            \frac{\partial y}{\partial r} &= \cos\phi \\
            \Big(\frac{\partial x}{\partial r}\Big)^2 + \Big(\frac{\partial y}{\partial r}\Big)^2 &= \sin^2\phi + \cos^2\phi = 1
          \end{align}
        \]
        Therefore equation \ref{eqn:first} is satisfied. To satisfy equation
        \ref{eqn:second} modify the equations to:
        \[
          \begin{align}
            x(r, \phi) &= r\sin(\phi\sin\alpha) \\
            y(r, \phi) &= r\cos(\phi\sin\alpha)
          \end{align}
        \]
        leaving the results of equation \ref{eqn:first} unchanged.
        \[
          \begin{align}
            \frac{\partial x}{\partial \phi} &= r\sin\alpha \cos(\phi\sin\alpha) \\
            \frac{\partial y}{\partial \phi} &= -r\sin\alpha \sin(\phi\sin\alpha) \\
            \Big(\frac{\partial x}{\partial \phi}\Big)^2 + \Big(\frac{\partial y}{\partial \phi}\Big)^2 &= r^2\sin^2\alpha\sin^2(\phi\sin\alpha) + r^2\sin^2\alpha\cos^2(\phi\cos\alpha)\\
                                                                                                        &= r^2\sin^2\alpha
          \end{align}
        \]
        Therefore \ref{eqn:second} is satisfied. Checking the last equation:
        \[
          \begin{align}
            \frac{\partial x}{\partial r}\frac{\partial x}{\partial \phi} &= -\frac{\partial y}{\partial r}\frac{\partial y}{\partial \phi} \\
            r\sin\alpha \sin(\phi\sin\alpha) \cos(\phi\sin\alpha) &= - (-r\sin\alpha \cos(\phi\sin\alpha) \sin(\phi\sin\alpha)) \\
          \end{align}
        \]
        Therefore the transformations that bring the metric into the form
        $ds^2 = dx^2 + dy^2$ are:
        \[
          \begin{align}
            x(r, \phi) &= r\sin(\phi\sin\alpha) \\
            y(r, \phi) &= r\cos(\phi\sin\alpha)
          \end{align}
        \]
        Proof:
        \[
          \begin{align}
            dx &= \sin(\phi\sin\alpha) dr + r\sin\alpha\cos(\phi\sin\alpha) d\phi\\
            dy &= \cos(\phi\sin\alpha) dr - r\sin\alpha\sin(\phi\sin\alpha) d\phi\\
          \end{align}
        \]
        and:
        \[
          \begin{align}
            ds^2 &= dx^2 + dy^2\\
                 &=\sin^2(\phi\sin\alpha)dr^2 + r^2\sin^2\alpha \cos^2(\phi\sin\alpha)d\phi^2 + 2r\sin\alpha\sin(\phi\sin\alpha)\cos(\phi\sin\alpha) dr d\phi\\
                 &+ \cos^2(\phi\sin\alpha)dr^2 +r^2\sin^2\alpha \sin^2(\phi\sin\alpha)d\phi^2 - 2rs\sin\alpha\sin(\phi\sin\alpha)\cos(\phi\sin\alpha) dr d\phi\\
                 &= dr^2 + r^2\sin^2\alpha d\phi^2
          \end{align}
        \]
        Consider the inverted transfromation:
        \[
          \begin{align}
            \phi &= \frac{1}{\sin\alpha} \arctan\Big(\frac{y}{x}\Big) \\
            r &= \sqrt{x^2 + y^2}
          \end{align}
        \]
        In order for these coordinates to cover the entire plane, the range
        of $\phi(x,y)$ and $r(x, y)$ must be $0 \leq \phi < 2\pi$ and
        $0 \leq r < \infty$. The range of $\arctan$ is $-\pi/2 < \phi < \pi / 2$.
        Check this later.
      \item \textbf{Prove that any vector parallel transported along a
        circle of constant $r$ on the surface of the cone ends up rotated
        by an angle $\beta$ after a complete trip. Express $\beta$ in terms
        of $\alpha$.}
        The equation for parallel transport is:
        \[
          t^a \nabla_a v^b = 0
        \]
        where $t^a$ is the tangent along the curve and $v^b$ is the vector
        being parallelly transported.
        \[
          t^a \partial_a v^b + t^a \Gamma^b_{ac} v^c = 0
        \]
        \[
          t^r \partial_r v^b + t^\phi \partial_\phi v^b +
          t^r \Gamma^b_{rr} v^r +t^r \Gamma^b_{r\phi} v^\phi + t^\phi \Gamma^b_{\phi r} v^r + t^\phi \Gamma^b_{\phi\phi } v^\phi = 0
        \]
        \[
          \Gamma^c_{ab} = \frac{1}{2} g^{cd} (\partial_a g_{bd} + \partial_b g_{ad} - \partial_d g_{ab} )
        \]
        \[
          \Gamma^c_{r\phi} = \frac{1}{2} g^{cd} (\partial_r g_{\phi d} + \partial_b g_{ad} - \partial_d g_{ab} )
        \]
    \end{enumerate}
\end{enumerate}
\end{document}
