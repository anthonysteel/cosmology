\documentclass[9pt]{report}
\usepackage{graphicx}
\usepackage[utf8]{inputenc}
%\usepackage[T1]{fontenc}
\usepackage{textcomp}
%\usepackage[dutch]{babel}
\usepackage{amsmath, amssymb}

\begin{document}
\title{A Relativist's Toolkit\protect\\ Problems}
\author{Anthony Steel}
\date{\today}
\maketitle
\chapter{Fundamentals}
\begin{enumerate}
  \item \textbf{The surface of a two-dimensional cone is embedded in
      three-dimensional flat space. The cone has an opening angle of
      $2\alpha$. Points on the cone which all have the same distance
      $r$ from the apex define a circle, and $\phi$ is the angle that
      runs along the circle.
    }
    \begin{enumerate}
      \item \textbf{Write down the metric of the cone, in terms of the
        coordinates $r$ and $\phi$.}

        The paramaterization of the cone $X(r, \phi)$ is:
        \[
          X(r, \phi) =
          \begin{bmatrix}
            r \sin\alpha \cos \phi\\
            r \sin\alpha \sin \phi \\
            r \cos \alpha
          \end{bmatrix}
        \]
        Differentiating with respect to $r$ and $\phi$ gives:
        \[
          X_{,\phi} =
          \begin{bmatrix}
            - r \sin\alpha \sin\phi \\
              r \sin\alpha \cos\phi \\
              0
          \end{bmatrix}
        \]
        and
        \[
          X_{,r} =
          \begin{bmatrix}
            \sin\alpha \cos \phi \\
            \sin \alpha \sin \phi \\
            \cos\alpha
          \end{bmatrix}
        \]
        The general form of the metric in two dimensions is:
        \[
          ds^2 = E d\phi^2 + 2F d\phi dr + G dr^2
        \]
        where:
        \[
          \begin{align}
            E &= X_{,\phi} \cdot X_{,\phi} = r\sin^2\phi \\
            F &= X_{,\phi} \cdot X_{,r} = 0\\
            G &= X_{,r} \cdot X_{,r} = 1
          \end{align}
        \]
        therefore:
        \[
          ds^2 = dr^2 + r\sin^2\phi d\phi^2
        \]

      \item \textbf{Find the coordinate transformation $x(r,\phi)$,
          $y(r,\phi)$ that brings the metric into the form $ds^2
        = dx^2+dy^2$. Do these coordinates cover the entire
        two-dimensional plane?}

        Consider:
        \[
        \begin{align}
          x(r,\phi) &= r\sin\alpha \cos\phi \\
          y(r,\phi) &= r\sin\alpha \sin\phi
        \end{align}
        \]
        Then:
        \[
          \frac{y}{x} = \tan\phi \to \phi = \arctan \frac{y}{x}
        \]
        and
        \[
          \begin{align}
            r = \frac{x}{\sin\alpha \cos\phi} = \frac{x \sqrt{(\frac{y}{x})^2 + 1}}{\sin\alpha} = \frac{\sqrt{x^2 + y^2 }}{\sin\alpha}
          \end{align}
        \]
        likewise:
        \[
          r = \frac{y}{\sin\alpha \sin\phi} = \frac{y  \sqrt{(\frac{y}{x})^2+1}}{\frac{y}{x}\sin\alpha} = \frac{\sqrt{x^2 + y^2}}{\sin\alpha}
        \]
        Proving that this transformation does indeed result in a metric of the
        form $ds^2 = dx^2 + dy^2$:
        \[
          \begin{align}
            d\phi &= \frac{\partial\phi}{\partial x} dx + \frac{\partial\phi}{\partial y} dy \\
                  &= \frac{-1}{1+\frac{x^2}{y^2}} dx + \frac{1}{1+\frac{y^2}{x^2}} dy \\
                  &= \frac{-y^2}{x^2+y^2}dx + \frac{x^2}{x^2+y^2}dy
          \end{align}
        \]
        \[
          d\phi^2 &= \frac{y^4}{(x^2+ y^2)^2} dx^2 + \frac{x^4}{(x^2+y^2)^2} dy^2 - 2 \frac{x^4y^4}{(x^2+y^2)^2}dx dy
        \]
        \[
        \sin^2\phi = \frac{y^2}{x^2 +y^2}
        \]
        \[
          \begin{align}
            dr &= \frac{\partial r}{\partial x} dx + \frac{\partial r }{\partial y} dy \\
               &= \frac{x}{\sin\alpha\sqrt{x^2 + y^2}} dx + \frac{y}{\sin\alpha \sqrt{x^2+y^2}}dy
          \end{align}
        \]
      \item \textbf{Prove that any vector parallel transported along a
        circle of constant $r$ on the surface of the cone ends up rotated
        by an angle $\beta$ after a complete trip. Express $\beta$ in terms
        of $\alpha$.}
    \end{enumerate}
\end{enumerate}
\end{document}
