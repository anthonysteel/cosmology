\documentclass[9pt]{report}
\usepackage{graphicx}
\usepackage[utf8]{inputenc}
\usepackage{textcomp}
\usepackage{amsmath, amssymb}

\begin{document}
\title{Chapter 2 Problems}
\author{Anthony Steel}
\date{\today}
\maketitle
\begin{enumerate}
\item
	\textbf{A tube of mass $M$ and length $l$ is free to swing around a pivot at
one end. A mass $m$ is positioned inside the (fricitonless) tube at this end.
The tube is held horizontal and then released. Let $\theta$ be the angle of the
tube with respect to the horizontal, and let $x$ be the distance the mass has
traveled along the tube. Find the Euler-Lagrange equations for $\theta$ and $x$,
and then write them in terms of $\theta$ and $\nu = x / l$ (the fraction of the
distance along the tube). These equations can only be solved numerically, and
you must pick a numerical value for the ratio $r = m / M$ in order to do this.
Write a prigram that produces the value of $\nu$ when the tube is vertical
$(\theta=\pi / 2)$}}

The total kinetic energy of the system will be the horizontal and
vertical translational kinetic energies of the mass $m$, and the rotational
kinetic energy of the tube.


Starting with the mass, consider its position $x'$ and $y'$:
\[
\begin{align}
x'(t) &= x(t) \sin \theta(t) \\
y'(t) &= x(t) \cos \theta(t) \\
\end{align}
\]
where $x$ is the position the mass has fallen down the tube. Differentiating
once and removing the explicit time dependence for brevity yields:
\[
\begin{align}
\dot{x}' &= \dot{x} \sin \theta + x \cos \theta \dot{\theta}\\
\dot{y}' &= \dot{x} \cos \theta - x \sin \theta \dot{\theta } \\
\end{align}
\]
Taking the squares of each:
\[
\begin{align}
(\dot{x}')^2 &= \dot{x}^2 \sin^2 \theta + x^2 \cos^2 \theta \dot{\theta}^2 + 2\sin \theta \cos \theta \dot{x} \dot{\theta} \\
(\dot{y}')^2 &= \dot{x}^2 \cos^2 \theta + x^2 \sin^2 \theta \dot{\theta }^2 - 2\sin \theta \cos \theta \dot{x} \dot{\theta} \\
\end{align}
\]
and adding:
\[
\begin{align}
(\dot{x}')^2 + (\dot{y}')^2 &= \dot{x}^2 \sin^2 \theta + \dot{x}^2 \cos^2 \theta + x^2 \cos^2 \theta \dot{\theta}^2 + x^2 \sin^2 \theta \dot{\theta }^2 + 2\sin \theta \cos \theta \dot{x} \dot{\theta} - 2\sin \theta \cos \theta \dot{x} \dot{\theta} \\
&=  \dot{x}^2 + x^2\dot{\theta}^2
\end{align}
\]
corresponding to radial and tangential kinetic energies respectively.
Therefore:
\[
T_m = \frac{1}{2} m (\dot{x}^2 + x^2\dot{\theta}^2)
\]
Now the moment of inertia for a tube rotating about its end is the same as
a rod rotating about its end, and is given by:
\[
I_M = \frac{1}{3} M l^2
\]
Therefore the total kinetic energies of the two masses are:
\[
T = \frac{1}{2} m (\dot{x}^2 + x^2\dot{\theta}^2) + \frac{1}{6}  M l^2 \dot{\theta}^2
\]
The potential energy for $m$ is simply:
\[
V_m = -mgl\cos\theta
\]
while the potnetial energy for $m$ is the vertical component of the position of
the center mass:
\[
V_M = -\frac{Mgl}{2}\cos\theta
\]
The Lagrangian $\mathcal{L}$ is:
\[
\mathcal{L} = T - V = \frac{1}{2} m (\dot{x}^2 + x^2\dot{\theta}^2) + \frac{1}{6}  M l^2 \dot{\theta}^2 + mgl\cos\theta + \frac{Mgl}{2}\cos\theta
\]
Finding the Euler-Lagrange equations for $x$:
\[
\begin{align}
\frac{d}{dt} \frac{\partial \mathcal{L}}{\partial\dot{x}} &= m \ddot{x}
\end{align}
\]
\[
\begin{align}
\frac{\partial \mathcal{L}}{\partial x} &= m x \dot{\theta}^2
\end{align}
\]
\[
\begin{align}
 \frac{d}{dt} \frac{\partial \mathcal{L}}{\partial\dot{x}} = \frac{\partial \mathcal{L}}{\partial x} \Rightarrow \ddot{x} &= x \dot{\theta}^2
\end{align}
\]
and $\theta$:
\[
\begin{align}
\frac{d}{dt} \frac{\partial \mathcal{L}}{\partial\dot{\theta}} &= m \frac{d}{dt}(x^2\dot{\theta}) + \frac{1}{3} Ml^2 \ddot{\theta}
\end{align}
\]
\[
\begin{align}
\frac{\partial \mathcal{L}}{\partial \theta} &= -gl (m + \frac{M}{2}) \sin \theta
\end{align}
\]
\[
\frac{d}{dt} \frac{\partial \mathcal{L}}{\partial\dot{\theta}} = \frac{\partial \mathcal{L}}{\partial \theta} \Rightarrow m \frac{d}{dt}(x^2\dot{\theta}) + \frac{1}{3} Ml^2 \ddot{\theta}
= -gl (m + \frac{M}{2}) \sin \theta
\]
Therefore the Euler Lagrange equations are:
\[
m \frac{d}{dt}(x^2\dot{\theta}) + \frac{1}{3} Ml^2 \ddot{\theta} = -gl (m + \frac{M}{2}) \sin \theta \label{el_theta}
\]
\[
\ddot{x} = x \dot{\theta}^2 \label{el_x}
\]
Reparameterizing in terms of $\eta$ and $r$:
\begin{equation}
r \frac{d}{dt}(\eta^2\dot{\theta}) + \frac{1}{3}\ddot{\theta} = -\frac{g}{l} (r + \frac{1}{2}) \sin \theta \label{el_eta}
\end{equation}
\begin{equation}
\ddot{\eta} = \eta \dot{\theta}^2 \label{el_theta}
\end{equation}
are the Euler-Lagrange equations.
\end{enumerate}
\end{document}
