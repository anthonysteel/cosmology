\documentclass[9pt]{report}
\usepackage{graphicx}
\usepackage[utf8]{inputenc}
\usepackage{textcomp}
\usepackage{amsmath, amssymb}

\begin{document}
\title{Chapter 2 Problems}
\author{Anthony Steel}
\date{\today}
\maketitle
\begin{enumerate}
\item
	\textbf{A tube of mass $M$ and length $l$ is free to swing around a pivot at
one end. A mass $m$ is positioned inside the (frictionless) tube at this end.
The tube is held horizontal and then released. Let $\theta$ be the angle of the
tube with respect to the horizontal, and let $x$ be the distance the mass has
traveled along the tube. Find the Euler-Lagrange equations for $\theta$ and $x$,
and then write them in terms of $\theta$ and $\nu = x / l$ (the fraction of the
distance along the tube). These equations can only be solved numerically, and
you must pick a numerical value for the ratio $r = m / M$ in order to do this.
Write a prigram that produces the value of $\nu$ when the tube is vertical
$(\theta=\pi / 2)$}}

\section{Derivation of Equations of Motion}

The total kinetic energy of the system will be the horizontal and
vertical translational kinetic energies of the mass $m$, and the rotational
kinetic energy of the tube.


Starting with the mass, consider its position $x'$ and $y'$:
\[
\begin{align}
x'(t) &= x(t) \sin \theta(t) \\
y'(t) &= x(t) \cos \theta(t) \\
\end{align}
\]
where $x$ is the position the mass has fallen down the tube. Differentiating
once and removing the explicit time dependence for brevity yields:
\[
\begin{align}
\dot{x}' &= \dot{x} \sin \theta + x \cos \theta \dot{\theta}\\
\dot{y}' &= \dot{x} \cos \theta - x \sin \theta \dot{\theta } \\
\end{align}
\]
Taking the squares of each:
\[
\begin{align}
(\dot{x}')^2 &= \dot{x}^2 \sin^2 \theta + x^2 \cos^2 \theta \dot{\theta}^2 + 2\sin \theta \cos \theta \dot{x} \dot{\theta} \\
(\dot{y}')^2 &= \dot{x}^2 \cos^2 \theta + x^2 \sin^2 \theta \dot{\theta }^2 - 2\sin \theta \cos \theta \dot{x} \dot{\theta} \\
\end{align}
\]
and adding:
\[
\begin{align}
(\dot{x}')^2 + (\dot{y}')^2 &= \dot{x}^2 \sin^2 \theta + \dot{x}^2 \cos^2 \theta + x^2 \cos^2 \theta \dot{\theta}^2 + x^2 \sin^2 \theta \dot{\theta }^2 + 2\sin \theta \cos \theta \dot{x} \dot{\theta} - 2\sin \theta \cos \theta \dot{x} \dot{\theta} \\
&=  \dot{x}^2 + x^2\dot{\theta}^2
\end{align}
\]
corresponding to radial and tangential kinetic energies respectively.
Therefore:
\[
T_m = \frac{1}{2} m (\dot{x}^2 + x^2\dot{\theta}^2)
\]
Now the moment of inertia for a tube rotating about its end is the same as
a rod rotating about its end, and is given by:
\[
I_M = \frac{1}{3} M l^2
\]
Therefore the total kinetic energies of the two masses are:
\[
T = \frac{1}{2} m (\dot{x}^2 + x^2\dot{\theta}^2) + \frac{1}{6}  M l^2 \dot{\theta}^2
\]
The potential energy for $m$ is simply:
\[
V_m = -mgx\cos\theta
\]
while the potnetial energy for $m$ is the vertical component of the position of
the center mass:
\[
V_M = -\frac{Mgl}{2}\cos\theta
\]
The Lagrangian $\mathcal{L}$ is:
\[
\mathcal{L} = T - V = \frac{1}{2} m (\dot{x}^2 + x^2\dot{\theta}^2) + \frac{1}{6}  M l^2 \dot{\theta}^2 + mgx\cos\theta + \frac{Mgl}{2}\cos\theta
\]
Finding the Euler-Lagrange equations for $x$:
\[
\begin{align}
\frac{d}{dt} \frac{\partial \mathcal{L}}{\partial\dot{x}} &= m \ddot{x}
\end{align}
\]
\[
\begin{align}
\frac{\partial \mathcal{L}}{\partial x} &= m x \dot{\theta}^2 + mg\cos\theta
\end{align}
\]
\[
\begin{align}
 \frac{d}{dt} \frac{\partial \mathcal{L}}{\partial\dot{x}} = \frac{\partial \mathcal{L}}{\partial x} \Rightarrow \ddot{x} &= x \dot{\theta}^2 + g \cos \theta
\end{align}
\]
and $\theta$:
\[
\begin{align}
\frac{d}{dt} \frac{\partial \mathcal{L}}{\partial\dot{\theta}} &= m \frac{d}{dt}(x^2\dot{\theta}) + \frac{1}{3} Ml^2 \ddot{\theta}
\end{align}
\]
\[
\begin{align}
\frac{\partial \mathcal{L}}{\partial \theta} &= -g (mx + \frac{Ml}{2}) \sin \theta
\end{align}
\]
\[
\frac{d}{dt} \frac{\partial \mathcal{L}}{\partial\dot{\theta}} = \frac{\partial \mathcal{L}}{\partial \theta} \Rightarrow m \frac{d}{dt}(x^2\dot{\theta}) + \frac{1}{3} Ml^2 \ddot{\theta}
= -g(mx + \frac{Ml}{2}) \sin \theta
\]
Therefore the Euler Lagrange equations are:
\[
m \frac{d}{dt}(x^2\dot{\theta}) + \frac{1}{3} Ml^2 \ddot{\theta} = -g (mx + \frac{Ml}{2}) \sin \theta \label{el_theta}
\]
\[
\ddot{x} = x \dot{\theta}^2 \label{el_x} + g \cos \theta
\]
Reparameterizing in terms of $\eta$ and $r=\frac{m}{M}$:
\begin{equation}
r \frac{d}{dt}(\eta^2\dot{\theta}) + \frac{1}{3}\ddot{\theta} = -\frac{g}{l} (\frac{rx}{l} + \frac{1}{2}) \sin \theta \label{el_eta}
\end{equation}
\begin{equation}
\ddot{\eta} = \eta \dot{\theta}^2 \label{el_theta} + \frac{g}{l}\cos \theta
\end{equation}
are the Euler-Lagrange equations.

\section{Dimensional analysis}
$\eta$ is dimensionless. The dimensional quantities in the problem are $m$,
$M$, $l$, $g$. To get a dimensonless number, the mass terms would have
to cancel in a ratio, but you could not cancel $l$ with $g$. You need to introduce
another quantity which has dimensions of time. Therefore we could also introduce
the frequency $\omega$ and write:
\[
  \begin{align}
    [\eta] &\propto [m]^a [M]^b [l]^c [g]^d [\omega]^e \\
           &\propto (\text{mass})^a (\text{mass})^b (\text{length})^c (\frac{\text{length}}{\text{time}^2})^d (\frac{1}{\text{time}})^e
  \end{align}
\]
Giving the equations:
\[
  \begin{align}
    a + b &= 0 \Rightarrow a = -b\\
    c + d &= 0 \Rightarrow c = -d\\
    -2d-e &= 0 \Rightarrow e = -2d
  \end{align}
\]
and implying:
\[
  \eta \propto \frac{m}{M}\frac{l}{g}\omega^2
\]
However, the frequency of an oscillator is:
\[
  \omega^2 \propto \frac{g}{l} f^2(\theta)
\]
So dimensional analysis would imply that:
\[
  \eta \propto \frac{m}{M}
\]
meaning that it should not depend on the length of the tube.

\section{Asymptotic Behaviour of Solutions}
Does the mass exit the tube? If $\eta \propto m / M$ is correct, then the mass
would exit the tube if $m > M$.

In this case, the mass exits the bottom of the tube and the system
essentially decouples. After that, the mass is simply in free fall in a
gravitational potential and the tube executes harmonic motion.

If the $\eta < 1$ when the tube is vertical, the mass never exits the tube.

\section{Approximations}
Though the equations of motion cannot be solved exactly, it would be nice to
determine whether $\eta(t_\text{vertical})$ depends on $l$ in some approximation.
To determine this we need an equation for $\eta$ by solving the $\eta$ equation
of motion.


The equations are coupled (i.e. $\theta$ appears in the equation of motion for
$\eta$ and $\eta$ appears in the equation of motion for $\theta$) so a useful
approximation would be one that decouples at least one of them.

\subsection{Small mass approximation $m \ll M$}
$m \ll M$ is equivalent to the condition $r \ll 1$. Let us assume that $r$ is
negligibly small. The equations of motion then read:
\begin{equation}
\ddot{\theta} = -\frac{3g}{2l} \sin \theta \label{small_r_el_theta}
\end{equation}
\begin{equation}
\ddot{\eta} = \eta \dot{\theta}^2 \label{small_r_el_eta} + \frac{g}{l} \cos \theta
\end{equation}
By our definition of $\theta$ the vertical position is at $\theta = 0$ so small
oscillations about the vertical are given by:
\[
\theta(t) = A \cos (\omega t + \phi)
\]
where $\omega^2 = 3 g / 2 l$.

Consider the initial conditions:
\[
  t_\text{vertical} = \frac{T}{2} = \frac{1}{2} \frac{2\pi}{\omega} = \frac{\pi}{\omega} = \frac{2\pi}{3}\sqrt{\frac{l}{g}}
\]
\[
 \theta(t_\text{vertical}) = 0
\]
Then:
\[
\begin{align}
0 &= \cos(\omega (\frac{\pi}{\omega}) + \phi) \\
  &= \cos(\pi + \phi)\\
  &= \cos(\pi)\cos(\phi)-\sin(\pi)\sin(\phi)\\
  &= \cos(\phi)
\end{align}
\]
therefore $\phi = \pi/2$ and the solution for $\theta$ can be rewritten without
a phase shift as:
\[
  \theta(t) = A \sin(\omega t)
\]
Taking a time derivative:
\[
\dot{\theta}(t) = A\omega \cos (\omega t)
\]
\[
\dot{\theta}^2(t) = A^2\omega^2 \cos^2 (\omega t)
\]
We are interested in the time $t = t_\text{vertical} + \delta$ where $\delta$ is
small. In this small region of time:
\[
  \theta (\delta) = A\omega \delta
\]
\[
  \dot{\theta} (\delta) = A \omega
\]
and:
\[
  \ddot{\eta} = A^2 \omega^2 \eta + \frac{g}{l}
\]
$a = -A^2 \omega^2$, $C_0 = \frac{g}{l}$.
\[
  \eta(t) = ( \frac{g}{-}
\]

\subsection{Taylor Series Expansion}
Taylor expand $\eta$ and $x$ as:
\[
  \eta(t) = \sum_{n=0}^\infty a_n (t-t_0)^n
\]
and:
\[
  x(t) = \sum_{n=0}^\infty b_n (t-t_0)^n
\]
Therefore:
\end{enumerate}
\end{document}
