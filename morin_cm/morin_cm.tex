\documentclass[9pt]{report}
\usepackage{graphicx}
\usepackage[utf8]{inputenc}
\usepackage{textcomp}
\usepackage{amsmath, amssymb}

\begin{document}
\title{Morin's Classical Mechanics}
\author{Anthony Steel}
\date{\today}
\maketitle
\chapter{}
\chapter{}
\chapter{}
\chapter{}
\chapter{}
\chapter{}
\chapter{}
\chapter{}
\begin{enumerate}
\item
	\textbf{A tube of mass $M$ and length $l$ is free to swing around a pivot at
one end. A mass $m$ is positioned inside the (frictionless) tube at this end.
The tube is held horizontal and then released. Let $\theta$ be the angle of the
tube with respect to the horizontal, and let $x$ be the distance the mass has
traveled along the tube. Find the Euler-Lagrange equations for $\theta$ and $x$,
and then write them in terms of $\theta$ and $\nu = x / l$ (the fraction of the
distance along the tube). These equations can only be solved numerically, and
you must pick a numerical value for the ratio $r = m / M$ in order to do this.
Write a prigram that produces the value of $\nu$ when the tube is vertical
$(\theta=\pi / 2)$}}

\section{Derivation of Equations of Motion}

The total kinetic energy of the system will be the horizontal and
vertical translational kinetic energies of the mass $m$, and the rotational
kinetic energy of the tube.


Starting with the mass, consider its position $x'$ and $y'$:
\[
\begin{align}
x'(t) &= x(t) \sin \theta(t) \\
y'(t) &= x(t) \cos \theta(t) \\
\end{align}
\]
where $x$ is the position the mass has fallen down the tube. Differentiating
once and removing the explicit time dependence for brevity yields:
\[
\begin{align}
\dot{x}' &= \dot{x} \sin \theta + x \cos \theta \dot{\theta}\\
\dot{y}' &= \dot{x} \cos \theta - x \sin \theta \dot{\theta } \\
\end{align}
\]
Taking the squares of each:
\[
\begin{align}
(\dot{x}')^2 &= \dot{x}^2 \sin^2 \theta + x^2 \cos^2 \theta \dot{\theta}^2 + 2\sin \theta \cos \theta \dot{x} \dot{\theta} \\
(\dot{y}')^2 &= \dot{x}^2 \cos^2 \theta + x^2 \sin^2 \theta \dot{\theta }^2 - 2\sin \theta \cos \theta \dot{x} \dot{\theta} \\
\end{align}
\]
and adding:
\[
\begin{align}
(\dot{x}')^2 + (\dot{y}')^2 &= \dot{x}^2 \sin^2 \theta + \dot{x}^2 \cos^2 \theta + x^2 \cos^2 \theta \dot{\theta}^2 + x^2 \sin^2 \theta \dot{\theta }^2 + 2\sin \theta \cos \theta \dot{x} \dot{\theta} - 2\sin \theta \cos \theta \dot{x} \dot{\theta} \\
&=  \dot{x}^2 + x^2\dot{\theta}^2
\end{align}
\]
corresponding to radial and tangential kinetic energies respectively.
Therefore:
\[
T_m = \frac{1}{2} m (\dot{x}^2 + x^2\dot{\theta}^2)
\]
Now the moment of inertia for a tube rotating about its end is the same as
a rod rotating about its end, and is given by:
\[
I_M = \frac{1}{3} M l^2
\]
Therefore the total kinetic energies of the two masses are:
\[
T = \frac{1}{2} m (\dot{x}^2 + x^2\dot{\theta}^2) + \frac{1}{6}  M l^2 \dot{\theta}^2
\]
The potential energy for $m$ is simply:
\[
V_m = -mgx\cos\theta
\]
while the potnetial energy for $m$ is the vertical component of the position of
the center mass:
\[
V_M = -\frac{Mgl}{2}\cos\theta
\]
The Lagrangian $\mathcal{L}$ is:
\[
\mathcal{L} = T - V = \frac{1}{2} m (\dot{x}^2 + x^2\dot{\theta}^2) + \frac{1}{6}  M l^2 \dot{\theta}^2 + mgx\cos\theta + \frac{Mgl}{2}\cos\theta
\]
Finding the Euler-Lagrange equations for $x$:
\[
\begin{align}
\frac{d}{dt} \frac{\partial \mathcal{L}}{\partial\dot{x}} &= m \ddot{x}
\end{align}
\]
\[
\begin{align}
\frac{\partial \mathcal{L}}{\partial x} &= m x \dot{\theta}^2 + mg\cos\theta
\end{align}
\]
\[
\begin{align}
 \frac{d}{dt} \frac{\partial \mathcal{L}}{\partial\dot{x}} = \frac{\partial \mathcal{L}}{\partial x} \Rightarrow \ddot{x} &= x \dot{\theta}^2 + g \cos \theta
\end{align}
\]
and $\theta$:
\[
\begin{align}
\frac{d}{dt} \frac{\partial \mathcal{L}}{\partial\dot{\theta}} &= m \frac{d}{dt}(x^2\dot{\theta}) + \frac{1}{3} Ml^2 \ddot{\theta}
\end{align}
\]
\[
\begin{align}
\frac{\partial \mathcal{L}}{\partial \theta} &= -g (mx + \frac{Ml}{2}) \sin \theta
\end{align}
\]
\[
\frac{d}{dt} \frac{\partial \mathcal{L}}{\partial\dot{\theta}} = \frac{\partial \mathcal{L}}{\partial \theta} \Rightarrow m \frac{d}{dt}(x^2\dot{\theta}) + \frac{1}{3} Ml^2 \ddot{\theta}
= -g(mx + \frac{Ml}{2}) \sin \theta
\]
Therefore the Euler Lagrange equations are:
\[
m \frac{d}{dt}(x^2\dot{\theta}) + \frac{1}{3} Ml^2 \ddot{\theta} = -g (mx + \frac{Ml}{2}) \sin \theta \label{el_theta}
\]
\[
\ddot{x} = x \dot{\theta}^2 \label{el_x} + g \cos \theta
\]
Reparameterizing in terms of $\eta = x / l $ and $r=\frac{m}{M}$:
\begin{equation}
r \frac{d}{dt}(\eta^2\dot{\theta}) + \frac{1}{3}\ddot{\theta} = -\frac{g}{l} (r\eta + \frac{1}{2}) \sin \theta \label{el_eta}
\end{equation}
\begin{equation}
\ddot{\eta} = \eta \dot{\theta}^2 \label{el_theta} + \frac{g}{l}\cos \theta
\end{equation}
are the Euler-Lagrange equations.

\section{Dimensional analysis}
$\eta$ is dimensionless. The dimensional quantities in the problem are $m$,
$M$, $l$, $g$. To get a dimensonless number, the mass terms would have
to cancel in a ratio, but you could not cancel $l$ with $g$. You need to introduce
another quantity which has dimensions of time. Therefore we could also introduce
the frequency $\omega$ and write:
\[
  \begin{align}
    [\eta] &\propto [m]^a [M]^b [l]^c [g]^d [\omega]^e \\
           &\propto (\text{mass})^a (\text{mass})^b (\text{length})^c (\frac{\text{length}}{\text{time}^2})^d (\frac{1}{\text{time}})^e
  \end{align}
\]
Giving the equations:
\[
  \begin{align}
    a + b &= 0 \Rightarrow a = -b\\
    c + d &= 0 \Rightarrow c = -d\\
    -2d-e &= 0 \Rightarrow e = -2d
  \end{align}
\]
and implying:
\[
  \eta \propto \frac{m}{M}\frac{l}{g}\omega^2
\]
However, the frequency of an oscillator is:
\[
  \omega^2 \propto \frac{g}{l} f^2(\theta)
\]
So dimensional analysis would imply that:
\[
  \eta \propto \frac{m}{M}
\]
meaning that it should not depend on the length of the tube.

\section{Asymptotic Behaviour of Solutions}
Does the mass exit the tube? If $\eta \propto m / M$ is correct, then the mass
would exit the tube if $m > M$.

In this case, the mass exits the bottom of the tube and the system
essentially decouples. After that, the mass is simply in free fall in a
gravitational potential and the tube executes harmonic motion.

If the $\eta < 1$ when the tube is vertical, the mass never exits the tube.

\section{Approximations}
Though the equations of motion cannot be solved exactly, it would be nice to
determine whether $\eta(t_\text{vertical})$ depends on $l$ in some approximation.
To determine this we need an equation for $\eta$ by solving the $\eta$ equation
of motion.


The equations are coupled (i.e. $\theta$ appears in the equation of motion for
$\eta$ and $\eta$ appears in the equation of motion for $\theta$) so a useful
approximation would be one that decouples at least one of them.

\subsection{Constant $\eta$}
Consider the effect of $\eta$ being constant (i.e
the mass has been fixed inside the tube). The equation of motion for $\theta$ is:
\[
  (r \eta^2 + \frac{1}{3}) \ddot{\theta} = -\frac{g}{l} (r \eta + \frac{1}{2}) \sin\theta
\]
This can be solved in the small angle approximation as:
\[
  \ddot{\theta} = - \frac{g}{l} \frac{r\eta + \frac{1}{2}}{r\eta^2 + \frac{1}{3}}\theta
\]
giving:
\[
  \theta(t) = A \cos \omega t
\]
where:
\[
  \omega^2 = \frac{g}{l} \frac{r\eta + \frac{1}{2}}{r\eta^2 + \frac{1}{3}}
\]

\subsection{Small mass approximation $m \ll M$}
$m \ll M$ is equivalent to the condition $r \ll 1$. Let us assume that $r$ is
negligibly small. The equations of motion then read:
\begin{equation}
\ddot{\theta} = -\frac{3g}{2l} \sin \theta \label{small_r_el_theta}
\end{equation}
\begin{equation}
\ddot{\eta} = \eta \dot{\theta}^2 \label{small_r_el_eta} + \frac{g}{l} \cos \theta
\end{equation}
By our definition of $\theta$ the vertical position is at $\theta = 0$ so small
oscillations about the vertical are given by:
\[
\theta(t) = A \cos (\omega t + \phi)
\]
where $\omega^2 = 3 g / 2 l$.

Consider the initial conditions:
\[
  t_\text{vertical} = \frac{T}{2} = \frac{1}{2} \frac{2\pi}{\omega} = \frac{\pi}{\omega} = \frac{2\pi}{3}\sqrt{\frac{l}{g}}
\]
\[
 \theta(t_\text{vertical}) = 0
\]
Then:
\[
\begin{align}
0 &= \cos(\omega (\frac{\pi}{\omega}) + \phi) \\
  &= \cos(\pi + \phi)\\
  &= \cos(\pi)\cos(\phi)-\sin(\pi)\sin(\phi)\\
  &= \cos(\phi)
\end{align}
\]
therefore $\phi = \pi/2$ and the solution for $\theta$ can be rewritten without
a phase shift as:
\[
  \theta(t) = A \sin(\omega t)
\]
Taking a time derivative:
\[
\dot{\theta}(t) = A\omega \cos (\omega t)
\]
\[
\dot{\theta}^2(t) = A^2\omega^2 \cos^2 (\omega t)
\]
We are interested in the time $t = t_\text{vertical} + \delta$ where $\delta$ is
small. In this small region of time:
\[
  \theta (\delta) = A\omega \delta
\]
\[
  \dot{\theta} (\delta) = A \omega
\]
and:
\[
  \ddot{\eta} = A^2 \omega^2 \eta + \frac{g}{l}
\]
$a = -A^2 \omega^2$, $C_0 = \frac{g}{l}$.
\[
  \eta(t) = ( \frac{g}{-}
\]

\subsection{Perturbation Series}
\[
r \frac{d}{dt}(\eta^2\dot{\theta}) + \frac{1}{3}\ddot{\theta} = -\frac{g}{l} (r\eta + \frac{1}{2}) \sin \theta
\]
\[
\ddot{\eta} = \eta \dot{\theta}^2 \label{el_theta} + \frac{g}{l}\cos \theta
\]
Take $\theta$ to be small and let $\epsilon = r$:
\[
\epsilon \frac{d}{dt}(\eta^2\dot{\theta}) + \frac{1}{3}\ddot{\theta} = -\frac{g}{l} (\epsilon \eta + \frac{1}{2}) \theta
\]
\[
\ddot{\eta} = \eta \dot{\theta}^2 \label{el_theta} + \frac{g}{l}
\]
Let:
\[
  \theta = \sum_{n = 0}^{\infty} a_n(t) \epsilon^n = a_0 + a_1 \epsilon
\]
\[
  \begin{align}
    \epsilon \frac{d}{dt}(\eta^2(\dot{a_0} + \dot{a_1}\epsilon)) + \frac{1}{3}(\ddot{a_0} + \ddot{a_1} \epsilon) &= -\frac{g}{l} (\epsilon \eta + \frac{1}{2}) (a_0 + a_1 \epsilon)  \\
  \frac{d}{dt}(\dot{a}_1\eta^2)\epsilon^2 +   ( \frac{d}{dt}(\dot{a}_0\eta^2)+ \ddot{a}_0 )\epsilon +  \frac{1}{3}\ddot{a}_0 &= -\frac{g}{l} (a_1 \eta \epsilon^2 + (a_0\eta + \frac{1}{2} a_1) \epsilon + \frac{1}{2}a_0)
  \end{align}
\]
Comparing terms of zeroth order:
\[
\ddot{a}_0 =  - \frac{3}{2} \frac{g}{l} a_0
\]
Therefore:
\[
  a_0 = A \cos\omega_0 t
\]
where $\omega_0^2 = 3 g / 2 l$.
Now comparing terms of first order:
\[
  \begin{align}
    \frac{d}{dt}(\dot{a}_0\eta^2)+ \ddot{a}_0 &= -\frac{g}{l}(a_0\eta + \frac{1}{2} a_1) \\
    a_1 &= \frac{l}{g} (\frac{d}{dt}(\dot{a}_0\eta^2)+ \ddot{a}_0 ) + a_0\eta \\
        &= \frac{l}{g} (\ddot{a}_0\eta^2+\dot{a}_0 \frac{d}{dt}(\eta^2)+ \ddot{a}_0 ) + a_0\eta  \\
        &= \frac{l}{g} (\ddot{a}_0(1 + \eta^2) +\dot{a}_0 \frac{d}{dt}(\eta^2)) + a_0\eta  \\
        &= \Big(-\frac{3}{2}+\eta -\frac{3}{2}\eta^2 + \omega_0 \frac{l}{g} \frac{d}{dt}(\eta^2) \Big)a_0
  \end{align}
\]
Simplifying $a_1^2$:
\[
  \begin{align}
    a_1^2 &= \Big(-\frac{3}{2}+\eta -\frac{3}{2}\eta^2 + \omega_0 \frac{l}{g} \frac{d}{dt}(\eta^2) \Big)^2 a_0^2 \\
          &= \Big(-\frac{3}{2}+\eta -\frac{3}{2}\eta^2 + \omega_0 \frac{l}{g} \frac{d}{dt}(\eta^2) \Big)^2 a_0^2 \\
          &\approx \Big(-\frac{3}{2}+\eta + \omega_0 \frac{l}{g} \frac{d}{dt}(\eta^2) \Big)^2 a_0^2 \\
  \end{align}
\]
\[
a_1^2 \approx (\frac{9}{4}-3\eta + 3 \omega_0 \frac{l}{g} \frac{d}{dt}(\eta^2))a_0^2
\]
Giving:
\[
  \theta(t) = a_0 + a_1 r
\]
Substituting into:
\[
  \begin{align}
    \ddot{\eta} &= \eta \dot{\theta}^2 + \frac{g}{l} \\
                &= \eta (a_0^2 + a_0 a_1 r + a_1^2 r^2) + \frac{g}{l} \\
                &= \eta (a_0^2 + \frac{3}{2} a_0^2 r + \frac{9}{4}a_0^2 r^2) + \frac{g}{l} \\
                &= \eta (1 + \frac{3}{2} r + \frac{9}{4} r^2) a_0^2 + \frac{g}{l} \\
  \end{align}
\]
\end{enumerate}
\chapter{}
\chapter{}
\chapter{Relativity (Kinematics)}
\begin{enumerate}
  \item \textbf{Effectively speed c}

  \textbf{A rocket flies between two planets that are one light-year-apart. What
  should the rocket's speed be so that time elapsed on the captain's watch is
  one year?}

  Let the time $t'$ be the time elapsed on the captain's watch and $t$ be the
  time observed by an external observer at rest with respect to the two planets.

  In this frame of reference:
  \[
  t = \gamma t' = \frac{t'}{\sqrt{1 - \frac{v^2}{c^2}}}
  \]
  The placement of $\gamma$ makes sense because the external observer will
  see the captain's watch running slower than the captain does.

  The time it takes $t$ for the rocket to travel between the planet will be
  the distance divided by the rocket's velocity:
  \[
    t = \frac{d}{v}
  \]
  Equating gives:
  \[
    \begin{align}
      \frac{d}{v} &= \frac{t'}{\sqrt{1 - \frac{v^2}{c^2}}} \\
      \frac{d^2}{v^2} &= \frac{t'^2}{1 - \frac{v^2}{c^2}} \\
      \frac{d^2}{v^2}(1-\frac{v^2}{c^2}) &= t'^2\\
      d^2&= t'^2v^2+\frac{v^2d^2}{c^2}\\
      \frac{d^2}{t'^2}&= v^2(1 +\frac{d^2}{t'^2c^2})\\
      v^2&= \frac{d^2}{t'^2}\frac{1}{1 +\frac{d^2}{t'^2c^2}}\\
    \end{aling}
  \]
  Remembering that:
  \[
    \frac{d}{t'} = \frac{c * \text{year}}{\text{year}} = c
  \]
  and substituting:
  \[
    v^2 = \frac{1}{2}c^2 \\
  \]
  \[
    v = \frac{1}{\sqrt{2}}c \\
  \]
  \item \textbf{A passing train}

  \textbf{A train of length $15$ cs moves at a speed of $3c/5$. How much time
  does it take to pass a person standing on the ground (as measured by that
  person)? Solve this by working in the frame of the person, and then again
  by working in the frame of the train}

  The person standing on the ground views the train to be length contracted.
  Let $l'$ be the proper length of the train and $l$ be the length observed
  by the person standing on the ground, then:
  \[
    l = \frac{l'}{\gamma}
  \]
  The time it takes for the train to pass is:
  \[
    t = \frac{l}{v} = \frac{l'}{v\gamma} = 15 \text{cs} * \frac{5}{3\text{c}} * \sqrt{1 - (\frac{3}{5})^2} = 20\text{s}
  \]
  Now in the frame of the train, the time it takes for the person to pass is:
  \[
    t' = \frac{l'}{v}
  \]
  The equation relating the time in the train frame $t'$ to the time in the
  person's frame $t$ is:
  \[
    t' = \gamma t \implies t = \frac{t'}{\gamma} = \frac{l'}{v\gamma} = 20\text{s}
  \]
  \item \textbf{Overtaking a train}

  \textbf{Train $A$ has length $L$. Train $B$ moves past $A$ (on a parallel
    track, facing the same direction) with relative speed $4c / 5$. The length
    of $B$ is such that $A$ says that the front of the train coincides at
    exactly the same time as the backs coincide. What is the time difference
    between the fronts coinciding and the backs coinciding, as measured by
    $B$?}

    In train $A$'s frame, train $B$ has been length contracted by a factor of
    $\gamma$. This means that the proper length of train $B$ will be larger
    than train $A$ by a factor $\gamma$.
    \[
      L_a = L \text{ and } L_b = \gamma L
    \]
    As train $A$ comes by in $B$'s frame, train $B$ views $A$ to be length
    contracted:
    \[
      L_a' = \frac{L}{\gamma}
    \]
    Therefore the difference in length between the two trains as seen in train
    B's frame is:
    \[
      \Delta L = \gamma L - \frac{L}{\gamma} = L (\gamma - \frac{1}{\gamma})
    \]
    and the time it takes train $A$ to pass through this distance is:
    \[
      \Delta t = \frac{\Delta L}{v} = \frac{L}{v}(\gamma - \frac{1}{\gamma}) = \frac{v}{c^2 \sqrt{1-\frac{v^2}{c^2}} }} L = \frac{4}{3c} L
    \]

  \item \textbf{Walking on a train}

    \textbf{A train of proper length $L$ and speed $3c / 5$ approaches a tunnel
      of length $L$. At the moment the front of the train enters the tunnel, a
      person leaves the front of the train and walks (briskly) toward the
      back. She arrives at the back of the train right when it (the back) leaves
      the tunnel.}

    \textbf{a) How much time does this take in the ground frame?}

    In the ground frame, the train is length contracted so that:
    \[
      L_\text{train} = \frac{L}{\gamma}
    \]
    and the distance for the end of the train to leave the tunnel
    is:
    \[
      L_\text{total} = L + L_\text{train} = (1 + \frac{1}{\gamma})L
    \]
    The time it takes is therefore:
    \[
      t = \frac{L_\text{total}}{v} = (1 + \frac{1}{\gamma}) \frac{L}{v} = \frac{3L}{c}
    \]
    \textbf{b) What is the person's speed with respect to the ground?}
    \[
      v_\text{person} = \frac{L}{t} = \frac{Lc}{3L} = \frac{c}{3}
    \]
    \textbf{c) How much time elapses on the person's watch?}

    Let $t'$ be the time observed in the ground frame, and $t$ be the time on
    the person watch. These two are related by:
    \[
      t' = \gamma t \implies t = \frac{t'}{\gamma} = \frac{3L}{c}\sqrt{1-\frac{1}{9}} = 2\sqrt{2}\frac{L}{c}
    \]
  \item \textbf{Simultaneous waves}

  \textbf{Alice flies past Bob at speed $v$. Right when she passes, they both
    set their watches to zero. When Alice's watch shows a time $T$, she waves
    to Bob. Bob then waves to Alice simultaneously (as measured by him) with Alice's
    wave (so this is before he actually sees her wave). Alice then
    waves to Bob simultaneously (as measured by her) with Bob's wave. Bob
    then waves to Alice simultaneously (as measured by him) with Alice's second
    wave. And so on. What are the readings on Alice's watch for all the time she
    waves? And likewise for Bob?}

    Consider the spacetime diagram in Alice's reference frame.
    Bob's worldline is given by:
    \[
      t = \frac{1}{\beta}x
    \]
    \[
      t = \beta x + T
    \]
    \[
      \frac{1}{\beta} x = \beta x + T
    \]
    \[
      x = \frac{1}{\frac{1}{\beta}-\beta} T
    \]
    To get the time:
    \[
      w_1 = (\frac{1}{\beta}x, x)
    \]
    \[
      w_2 = (\beta x+T, x)
    \]
    \[
      w_1 \cdot w_2 = -\frac{1}{\beta}x(\beta x + T) + x^2 = 0
    \]
\end{enumerate}
\end{document}
