\documentclass[9pt]{report}
\usepackage{graphicx}
\usepackage{bm}
\usepackage[utf8]{inputenc}
%\usepackage[T1]{fontenc}
\usepackage{textcomp}
%\usepackage[dutch]{babel}
\usepackage{amsmath, amssymb}

\begin{document}
\title{Srednicki's Quantum Field Theory\protect\\ Problems}
\author{Anthony Steel}
\date{\today}
\maketitle
\chapter{Attempts at relativistic quantum mechanics}
\begin{enumerate}
  \item
  \textbf{Show that the Dirac matrices must be even dimensional. Hint: show that
  the eigenvalues of $\beta$ are all $\pm1$, and that $\text{Tr}\beta = 0$, consider,
e.g., $\text{Tr}\alpha^2_1\beta$. Similarly, show that $\text{Tr}\alpha_i=0$}

  To determine the eigenvalues of $\beta$ consider the second commutation relation:
  \[
    (\beta^2)_{ab} = \delta_{ab} \to (\beta)_{ab} = \pm \sqrt{\delta_{ab}}  = \pm \delta_{ab}
  \]
  Therefore the eigenvalues of $\beta$ are $\pm 1$.
  Now to show $\text{Tr}\beta=0$, three relationships are needed. Consider:
  \[
    \begin{align}
    \{\alpha^j,\alpha^k\}_{ab} &= 2\delta^{jk}\delta_{ab} \\
    \end{align}
  \]
  In the case where $\alpha^j = \alpha^k$ and $a = b = 1$:
  \begin{equation}
    \begin{align}
    \alpha^2_1 = 1\\
    \end{align}
  \end{equation}
  Secondly consider that the trace is cyclic. That is:
  \begin{equation}
    \begin{align}
    \text{Tr}AB &= \text{Tr}BA\\
    \end{align}
  \end{equation}
  Finally using the commutation relation:
  \[
    \begin{align}
      \{\alpha^j, \beta\}_{ab} &= 0\\
      (\alpha^j\beta + \beta\alpha^j)_{ab} &= 0\\
      (\alpha^j\beta)_{ab} &= -(\beta\alpha^j)_{ab}\\
    \end{align}
  \]
  Substituting and summarizing:
  \[
    \begin{align}
      \text{Tr}\beta&= \text{Tr}\beta(1)\\
                    &= \text{Tr}\beta\alpha_1^2\\
                    &= \text{Tr}\beta\alpha_1\alpha_1\\
                    &= \text{Tr}\alpha_1\beta\alpha_1\\
                    &= -\text{Tr}\beta\alpha_1\alpha_1\\
                    &= -\text{Tr}\beta\alpha_1^2\\
                    &= -\text{Tr}\beta\\
    \end{align}
  \]
  The only number where $\text{Tr}\beta = -\text{Tr}\beta$ is $0$. Therefore:
  \[
    \text{Tr}\beta = 0
  \]
  Similarily to show that $\text{Tr}\alpha_i = 0$, consider $\beta^2_1 = 1$.
  \[
    \begin{align}
      \text{Tr}\alpha_i &= \text{Tr}\alpha_i(1)\\
                        &= \text{Tr}\alpha_i\beta^2_1\\
                        &= \text{Tr}\beta_1\alpha_i\beta_1\\
                        &= -\text{Tr}\alpha_i\beta_1\beta_1\\
                        &= -\text{Tr}\alpha_i\beta_1^2\\
                        &= -\text{Tr}\alpha_i\\
    \end{align}
  \]
  Again this relationship is only true for $\text{Tr}\alpha_i = 0$.

\item \textbf{With the hamiltonian of eq. (1.32), show that the state defined in
  eq. (1.33) the abstract Schr\"{o}dinger equation, eq. (1.1), if and only if
  the wave function obeys eq. (1.30). Your demonstration should apply both to the
  case of bosons, where the particle creation and annihilation operators obey
  the commutation relations of eq. (1.31), and to fermions, where the particle
  creation and annihilation operators obey the anticommutation relations of
  eq. (1.38)}
  \[
    \begin{align}
      H = &\int d^3 x a^\dagger(\bm{x})\Big(-\frac{h^2}{2m} \nabla^2 + U(\bm{x}) \Big) a(\bm{x}) \\
    &+ \frac{1}{2} \int d^3xd^3y V(\bm{x} - \bm{y}) a^\dagger(\bm{x}) a^\dagger(\bm{y}) a(\bm{x})a(\bm{y})
    \end{align}
  \]

\item \textbf{Show explicitly that $[N, H] = 0$, where $H$ is given by eq.
  (1.32) and $N$ by eq. (1.35).}
  \[
    \begin{align}
      H = &\int d^3 x a^\dagger(\bm{x})\Big(-\frac{h^2}{2m} \nabla^2 + U(\bm{x}) \Big) a(\bm{x}) \\
    &+ \frac{1}{2} \int d^3xd^3y V(\bm{x} - \bm{y}) a^\dagger(\bm{x}) a^\dagger(\bm{y}) a(\bm{x})a(\bm{y})
    \end{align}
  \]
  \[
    N = \int d^3x a^\dagger(\bm{x})
  \]

\end{enumerate}
  \chapter{Lorentz invariance}
  \begin{enumerate}
    \item \textbf{Verify that eq. (2.8) follows from eq. (2.3).}

      Consider eq (2.3) given by:
      \[
        g_{\mu\nu} \Lambda^\mu_\rho \Lambda^\nu_\sigma = g_{\rho\sigma}
      \]
      Writing the Lorentz transformation in terms of the infinitesimal
      Lorentz transformation gives:
      \[
        \begin{align}
        g_{\mu\nu} \Lambda^\mu_\rho \Lambda^\nu_\sigma &= g_{\rho\sigma}\\
        g_{\mu\nu} (\delta^\mu_\rho + \delta\omega^\mu_\rho)(\delta^\nu_\sigma+\delta\omega^\nu_\sigma) &= g_{\rho\sigma}\\
        g_{\mu\nu} (\delta^\mu_\rho \delta^\nu_\sigma + \delta^\mu_\rho \delta\omega^\nu_\sigma+\delta\omega^\mu_\rho\delta^\nu_\sigma+\delta\omega^\mu_\rho\delta\omega^\nu_\sigma) &= g_{\rho\sigma}\\
        g_{\mu\nu}\delta^\mu_\rho \delta^\nu_\sigma + g_{\mu\nu}\delta^\mu_\rho \delta\omega^\nu_\sigma+g_{\mu\nu}\delta\omega^\mu_\rho\delta^\nu_\sigma+g_{\mu\nu}\delta\omega^\mu_\rho\delta\omega^\nu_\sigma &= g_{\rho\sigma}\\
        g_{\rho\nu}\delta^\nu_\sigma + g_{\rho\nu} \delta\omega^\nu_\sigma+g_{\mu\sigma}\delta\omega^\mu_\rho+O(\delta\omega^2)&= g_{\rho\sigma}\\
        \end{align}
      \]
      Because $\delta\omega$ is an infinitesimal we ignore terms appearing quadratic
      in it. Therefore:
      \[
        \begin{align}
        g_{\rho\nu}\delta^\nu_\sigma + g_{\rho\nu} \delta\omega^\nu_\sigma+g_{\mu\sigma}\delta\omega^\mu_\rho&= g_{\rho\sigma}\\
        g_{\rho\sigma}+ \delta\omega_{\rho\sigma}+\delta\omega_{\sigma\rho}&= g_{\rho\sigma}\\
        \delta\omega_{\rho\sigma}+\delta\omega_{\sigma\rho}&= 0\\
        \delta\omega_{\rho\sigma}&=-\delta\omega_{\sigma\rho}\\
        \end{align}
      \]
      Thereby verifying eq (2.8):
      \[
          \delta\omega_{\rho\sigma} = -\delta\omega_{\sigma\rho}
      \]
  \end{enumerate}
\end{document}
